\let\negmedspace\undefined
\let\negthickspace\undefined
\documentclass{article}
\usepackage{cite}
\usepackage{amsmath,amssymb,amsfonts,amsthm}
\usepackage{algorithmic}
\usepackage{graphicx}
\usepackage{textcomp}
\usepackage{xcolor}
\usepackage{txfonts}
\usepackage{listings}
\usepackage{enumitem}
\usepackage{mathtools}
\usepackage{gensymb}
\usepackage[breaklinks=true]{hyperref}
\usepackage{tkz-euclide} % loads  TikZ and tkz-base
\usepackage{listings}
\usepackage{gvv}
%
%\usepackage{setspace}
%\usepackage{gensymb}
%\doublespacing
%\singlespacing

%\usepackage{graphicx}
%\usepackage{amssymb}
%\usepackage{relsize}
%\usepackage[cmex10]{amsmath}
%\usepackage{amsthm}
%\interdisplaylinepenalty=2500
%\savesymbol{iint}
%\usepackage{txfonts}
%\restoresymbol{TXF}{iint}
%\usepackage{wasysym}
%\usepackage{amsthm}
%\usepackage{iithtlc}
%\usepackage{mathrsfs}
%\usepackage{txfonts}
%\usepackage{stfloats}
%\usepackage{bm}
%\usepackage{cite}
%\usepackage{cases}
%\usepackage{subfig}
%\usepackage{xtab}
%\usepackage{longtable}
%\usepackage{multirow}
%\usepackage{algorithm}
%\usepackage{algpseudocode}
%\usepackage{enumitem}
%\usepackage{mathtools}
%\usepackage{tikz}
%\usepackage{circuitikz}
%\usepackage{verbatim}
%\usepackage{tfrupee}
%\usepackage{stmaryrd}
%\usetkzobj{all}
%    \usepackage{color}                                            %%
%    \usepackage{array}                                            %%
%    \usepackage{longtable}                                        %%
%    \usepackage{calc}                                             %%
%    \usepackage{multirow}                                         %%
%    \usepackage{hhline}                                           %%
%    \usepackage{ifthen}                                           %%
  %optionally (for landscape tables embedded in another document): %%
%    \usepackage{lscape}     
%\usepackage{multicol}
%\usepackage{chngcntr}
%\usepackage{enumerate}

%\usepackage{wasysym}
%\documentclass[conference]{IEEEtran}
%\IEEEoverridecommandlockouts
% The preceding line is only needed to identify funding in the first footnote. If that is unneeded, please comment it out.

\newtheorem{theorem}{Theorem}[section]
\newtheorem{problem}{Problem}
\newtheorem{proposition}{Proposition}[section]
\newtheorem{lemma}{Lemma}[section]
\newtheorem{corollary}[theorem]{Corollary}
\newtheorem{example}{Example}[section]
\newtheorem{definition}[problem]{Definition}
%\newtheorem{thm}{Theorem}[section] 
%\newtheorem{defn}[thm]{Definition}
%\newtheorem{algorithm}{Algorithm}[section]
%\newtheorem{cor}{Corollary}
\newcommand{\BEQA}{\begin{eqnarray}}
\newcommand{\EEQA}{\end{eqnarray}}
\theoremstyle{remark}
\newtheorem{rem}{Remark}

%\bibliographystyle{ieeetr}
\begin{document}
\title{Latex Assignment13}
\author{D.V.S. NIKHIL}
\date{28 August, 2023}
\maketitle
\section*{Ex 11.11.1}
In each of the following exercise \ref{prob:1} to \ref{prob:5}, find the equation of the circle with:
\begin{enumerate}[label=\arabic*.,ref=\thesubsection.\theenumi]
\item centre $(0,2)$ and radius $2$ \label{prob:1}
\item centre $(-2,3)$ and radius $4$
\item centre $\frac({1}{2},\frac{1}{4})$ and radius $\frac {1}{!2}$
\item centre $(1,1)$ and radius $2$
\item centre $(-a,-b)$ and radius $\sqrt{a^2-b^2}$  \label{prob:5}
\end{enumerate}
In each of the following exercise \ref{prob:6} to \ref{prob:9}, find the centre and radius of the circles
\begin{enumerate}[resume]
\item $(x-5)^2+(y-3)^2=36$ \label{prob:6}
\item $x^2+y^2-4x-8y-45=0$
\item $x^2+y^2-8x+10y-12=0$
\item $2x^2+2y^2-x=0$ \label{prob:9}
\end{enumerate}
\begin{enumerate}[resume]
\item Find the equation of the circle passing through the points $(4,1)$ and $(6,5)$ and whose centre is on the line $4x+y=16$.
\item Find the equation of the circle passing through the points $(2,3)$ and $(-1,1)$ and whose centre is on the line $x-3y-11=0$.
\item Find the equation of the circle with radius 5 whose centre lies on x-axis and passes through the point $(2,3)$.
\item Find the equation of the circle passing through $(0,0)$ and making intercepts $a$ and $b$ on the coordinate axes.
\item Find the equation of a circle with centre $(2,2)$ and passes through the point $(4,5)$.
\item Does the point $(-2.5,3.5)$ lie inside, outside or on the circle $x^2+y^2=25$.
\end{enumerate}
\end{document}

